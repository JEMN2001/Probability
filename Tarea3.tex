\documentclass[14pt]{article}
\usepackage{amssymb}
\usepackage[spanish]{babel}

\begin{document}
	\title{Tarea 3}
	\author{Juan Murcia, Samuel Perez, Nicolas Duque}
	\maketitle
	
	\begin{enumerate}
		\item \begin{enumerate}
			\item $M_x = e^{\mu_{x}s+\frac{\sigma_{x}^2s^2}{2}}$\\
			$M_y = e^{\mu_{y}s+\frac{\sigma_{y}^2s^2}{2}}$\\
			$M_z = M_x\cdot M_y = e^{(\mu_x+\mu_y)s+\frac{(\sigma_x^2+\sigma_y^2)s^2}{2}}$\\\\
			Luego Z es una variable aleatoria normal, con $\mu_z=\mu_x+\mu_y$ y $\sigma_z^2=\sigma_x^2+\sigma_y^2$\\
		\item En el código adjunto
		\item En el código adjunto\\
		\end{enumerate}
	\item \begin{enumerate}
		\item $f(x_i) = (1-p)^{x_i-1}p~~;~~f(n)=(1-q)^{n-1}q~~;~~Z=\sum_{i=1}^{N}X_i$
		$$M_n(s)=\sum_{x=1}^{n}e^{sx}(1-p)^{x-1}p$$
		$$= e^s\sum_{x=1}^{n}(e^s(1-p))^{x-1}p$$
		(Por serie geometrica)$$ = \frac{pe^s}{1-e^s(1-p)}$$
		$$= \frac{p}{e^{-s}-1+p}$$
		Luego $M_{x_i}=\frac{p}{e^{-s}-1+p},M_n=\frac{q}{e^{-s}-1+q}$\\
		$$M_z=M_n(ln(M_{x_i}))=\frac{q}{\frac{e^{-s}-1+p}{p}-1+q}
		$$
		$$= \frac{qp}{e^{-s}-1+qp}$$
		Luego Z es una variable aleatoria geometrica, con parametro $pq$.
		\item En el código adjunto
		\item En el código adjunto\\
	\end{enumerate}
	\item En el códico adjunto
	\end{enumerate}
\end{document}
